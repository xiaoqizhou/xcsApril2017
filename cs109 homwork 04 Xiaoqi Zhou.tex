%CS 109 Problem Set Template
%B. E. Burr

\documentclass{article}
	% basic article document class
	% use percent signs to make comments to yourself -- they will not show up.

\usepackage{amsmath}
\usepackage{amssymb}
	% packages that allow mathematical formatting

\usepackage{graphicx}
	% package that allows you to include graphics

\usepackage[top=1in, bottom=1in, left=1in, right=1in]{geometry}

\frenchspacing
	% one space after periods

\usepackage{fancyhdr}
	% allows custom headers

\pagestyle{fancy}

\lhead{CS 109, Stanford University \\ Problem Set 04} 
\rhead{Xiaoqi Zhou (xqzhou@stanford.edu) \\ 06237147}

\cfoot{\thepage}
\renewcommand{\footrulewidth}{0.4pt} 
	%footer
\usepackage{color}
\begin{document}
\thispagestyle{fancy} %shows header/footer

\newcommand{\myansw}{\textbf{Answer:}\\}

\newcommand{\mysolu}{\textbf{Solution:}\\}

\begin{enumerate}


	\item 
	
	\begin{enumerate}
		% 1.a
		\item
		\mysolu
		There are only 2 possible cases while picking up 2 from total k prime numbers: different or same. So we can add the counts of the above 2 cases as the total combinations.\\
		\myansw
		There are \\
		\colorbox{yellow}{
			${ {k \choose 2} + {k \choose 1} = \frac{k!}{2!(k-2)!}+k=\frac{k(k-1)}{2}+k}$
		}\\
		ways of choosing 2 prime numbers less than N.\\
			
		%1.b
		\item
		\mysolu
		To solve this problem, we can create a model which has ${r}$ idential divider and k sorted items (prime numbers). The number just after each divider would be considered as one of the selected numbers. The  First part of the polynomial represents the counts of the items + dividers combination. Second part represents the counts of the cases that at least one divider locates at the end of the queue.\\
		\myansw
		There are\\
		\colorbox{yellow}{
			${  {k+r \choose k,r}  - {k+(r-1) \choose k, (r-1)}= \frac{(k+r)!}{r!k!}-\frac{(k+r-1)!}{(r-1)!k!} = \frac{k(k+r-1)!}{r!k!}}$
		}\\
		ways of choosing ${r}$ prime numbers less than N.\\
		
		%1.c
		\item
		\mysolu
		Since the prime numbers set only includes distinct items in the beginning. We can just use combination of ${k}$ prime numbers taken ${r}$ at a time.\\
				\myansw
		There are\\
		\colorbox{yellow}{
			${{k \choose r}=\frac{k!}{(k-r)!r!}}$
		}\\
		ways of choosing ${r}$ distinct prime numbers less than N.\\

	\end{enumerate}
	
	\item
	\begin{enumerate}
		%2.a
		
		\item 
		\mysolu
		This is a permutation of all the 26 distinct letters.\\
		\myansw
		There are\\
		\colorbox{yellow}{
			${26!}$
		}\\
		ways for the 26 letters to be ordered if each letter appears exactly once and there are no other restrictions.\\
		
		%2.b
		\item
		\mysolu
		We can treat Q and U as a bundle and then the solution is the permutation of 25 items (24 letters and 1 bundle) multiply by permutation of 2 letters which are close to each other.\\
		ways for the 26 letters to be ordered if each letter appears exactly once and the letters Q and U must be next to each other (but in any order).\\
				\myansw
		There are\\
		\colorbox{yellow}{
			${25!2!}$
		}\\
		ways for the 26 letters to be ordered if each letter appears exactly once and the letters Q and U must be next to each other (but in any order).\\
		
		
		%2.c
		\item
		\mysolu
		There are ${26-5+1}$ gaps in queue of all the consonants, the 5 vowels can be considered 5 dividers been inerst into those gaps but one vowel at most in each gap. There are ${{26-5+1 \choose 5}5!}$ ways to select and permute 5 gaps to fit 5 vowels. The final result will be the permutation of vowels multiply by the permutation of consonants ${(26-5)!}$. \\
		\myansw
		There are\\
		\colorbox{yellow}{
			${{26-5+1 \choose 5}5!(26-5)! ={22 \choose 5}5!(26-5)! = \frac{22!}{17!5!}5!21! = \frac{22!21!}{17!}}$
		}\\
		ways for the 26 letters to be ordered if each letter appears exactly once and no two vowels can be next to each other.\\
		
		%2.d
		\item
		\mysolu
		We can make all the 5 vowels as a group and the permutation of the 26 consonants with that group is ${{(26-5+1)!}}$. The permutation within the vowels group is ${5!}$.\\
		\myansw
		There are\\
		\colorbox{yellow}{
			${(26-5+1)!5!}$
		}\\
		ways for the 26 letters to be ordered if each letter appears exactly ones and 5 vowels must be next to each other. 
		
		%2.e
		\item
		\mysolu
		If the position of the three most common letters are fixed. For each case, for example, "E" at position 1, "A" at position 2 and "T" at position 3. The number of the permutation is decided by the permutation of the ${26 - 3 = 23}$ rest letters.
		\myansw
		There are\\
		\colorbox{yellow}{
			${(26-3)! = 23!}$
		}\\
		ways for the 26 letters to be ordered if the position of the three most coomon letters (E,T and A) are fixed.\\
		
		
	\end{enumerate}

	\item
	\begin{enumerate}
		%3.a
		\item
		\mysolu
		For each distinct card, there are 4 duplications. The answer should be the number of possible divisions of 208 total positions into 52 distinct groups of size 4.\\
		\myansw
		There are\\
		\colorbox{yellow}{
			${\frac{208!}{(4!)^{52}}}$
		}\\
		distinct ways for the cards to be ordered.\\
		
		%3.b
		\item
		\mysolu
		There a 2 mutual exclusive cases for the combination of 2 cards, 2 different cards or same cards. For the 2 different cards case, there are ${{52 \choose 2}}$ ways of being dealt 2 cards. For the same cards case, there are just 52 ways.\\
		\myansw
		There are\\
		\colorbox{yellow}{
			${{52 \choose 2}+52 = \frac{52!}{2!50!}+52}$
		}\\
		distinct ways of being dealt two cards.
		
		%3.c
		\item
		\mysolu
		There are only 5 good cards and the combination of them are very limited. Just as the previous question, we can consider 2 cases and the total number of choice is reduced from 52 to 5.
		\myansw
		There are\\
		\colorbox{yellow}{
			${{5 \choose 2}+5 = 10 + 5 = 15}$
		}\\
		ways for you to get two "good" cards.
		
	\end{enumerate}
	\item
	\begin{enumerate}
		%4.a
		\item
		\mysolu
		To arrive the destination, the robot must move to the up ${n-1}$ times and move to the right $(m-1)$ times. Then the question can be an equivalence to a permutation of $(n-1)$ "move to the up" items and ${m-1}$ "move to the right" items.\\
		\myansw
		There are\\
		\colorbox{yellow}{
			${{(n-1)+(m-1) \choose (n-1),(m-1)} = \frac{(n+m-2)!}{(n-1)!(m-1)!}}$
		}\\
		distinct paths for the robot to take to the desination in cell ${(n,m)}$.\\
		
		%4.b
		\item
		\mysolu
		If the first step of the robot must be moving to the right, then the number of paths equals to the case with ${(n,m-1)}$ grid.\\
		\myansw
		There are\\
		\colorbox{yellow}{
			${{(n-1)+(m-2) \choose (n-1),(m-2)} = \frac{(n+m-3)!}{(n-1)!(m-2)!}}$
		}\\
		distinct paths for the robot to take to the destination in cell ${(n,m)}$, if it must start by moving to the right.\\
		
		%4.c
		\item
		\mysolu
		There are 2 possible directions for the robot at the beginning, up or right. We can analyze the case that the robot start by moving to the right. Since the total direction change number is odd, that means the last direction change must be "right to up" and happens in the last column. In any cases, there is no other path but go straight to the destination after the last turn. So we can just consider the number of the combinations of the first 2 turns. The first turn must happen in one of the ${m-2}$ columns, which exclude the first and last column; the second turn must happen in one of the ${n-2}$ rows, which also exclude the first and last row. Then the total combinations of possible paths are ${(m-2)(n-2)}$. The situation of the case that the robot start by moving to the up is similar.\\
		\myansw
		There are\\
		\colorbox{yellow}{
			${(m-2)(n-2)+(n-2)(m-2)=2(m-2)(n-2)}$
		}\\
		distinct paths for the robot to take to destination in cell ${(n,m)}$, if the robot changes direction exactly 3 times.\\
		
		
	\end{enumerate}
	\item
	\begin{enumerate}
		%5.a
		\item
		\mysolu
		Since there are minimal investments for each company and each company must be invested, \textdollar1, \textdollar2, \textdollar3, \textdollar4 million can be assign to the respective company at first and then work out strategies based on rest of the \textdollar10 million. Each \textdollar1 million can choose any one of the total 4 companies.
		\myansw
		There are\\
		\colorbox{yellow}{
			${(20-10)^{4}}$
		}\\
		different investment strategies are available if an investment must be made in each company.\\
		
		%5.b
		\item 
		\mysolu
		We can consider 5 cases including 4 different cases that there is one distinct company have no investment and one more case that all the companies have investment.
		\myansw
		There are\\
		\colorbox{yellow}{
			${(20-10)^{4}+\sum\limits_{i=6}^9(20-i)^3}$
		}\\
		different investment strategies are available if investments must be made in at least 3 of the 4 companies.\\
		
		
		 
	\end{enumerate}
	\item
	%6

		\mysolu
		Define summation of all the vectors ${m=\sum\limits_{i=1}^nX_i}$. Since ${m\leq{k}}$, there are ${k+1}$ possible value of ${m}$. For each distinct ${m}$, we can consider there are ${n-1}$ dividers insert into ${m}$ items. The total possible combination of particular ${m}$ is ${{m+n-1\choose n-1,m+1}={ m+n-1 \choose n-1,m}}$. Then for all the possible ${m}$, we can have ${\sum\limits_{i=0}^k{ m+n-1 \choose n-1,m}}$ different vectors.\\
		\myansw
		The number of vectors is\\
		\colorbox{yellow}{
			${\sum\limits_{i=0}^k{ m+n-1 \choose n-1,m}=\sum\limits_{i=0}^k \frac{(m+n-1)!}{(n-1)!(m)!}}$
		}\\
		

	\item
	%7

		\mysolu
		The total number of requests which can be \textquotedblleft pre-assigned\textquotedblright is ${k = \sum\limits_{i=1}^i m_i}$. The rest of identical requests are distributed to ${r}$ servers and the number of ways can be calculated by using ${(r-1)}$ dividers inserted into the queue of ${(n-k)}$ requests.\\
		\myansw
		There are\\
		\colorbox{yellow}{
			${{n-k+r-1 \choose n-k,r-1}=\frac{(n-k+r-1)!}{(n-k)!(r-1)!}}$ where ${k = \sum\limits_{i=1}^i m_i}$
		}\\
		ways for the requests to be distributed.\\



	\item
	\begin{enumerate}
		%8.a
		\item
		\mysolu
		Sample set is\\
		${S=\{{\frac{52!}{47!}}\text{ permutations of 5 cards from total 52}\}}$\\
		
		For each suit, the event set is\\
		${E_n=\{{\frac{13!}{8!}}\text{ permutations of 5 cards from total 13 cards in same suit}\}}$\\
		and all the 4 event set are exclusive.\\
		\myansw
		The probability of being dealt a flush is\\
		\colorbox{yellow}{
			${
				P(E)=\frac{|E_1\bigcup E_2\bigcup E_3\bigcup E_4|}{|S|} = \frac{(13\times14\times12\times11\times10\times9)\times4}{52\times51\times50\times49\times48}=\frac{617760}{311875200}=0.00198
			}$
		}\\
		
		%8.b
		\item
		\mysolu
		For the number of elements of the event ${E=\{\text{one pair in 5 cards}\}}$, we can select 1 numeric value from 13 numbers and pick up 2 cards in the total 4. Then we can add another 3 cards from the rest 48 cards. After all the 5 cards are selected, we can calculate the permutation of the 5 cards.\\
		${|E|=13\cdot{4 \choose 2}\cdot{48 \choose 3} \cdot 5!}$\\
		\myansw
		The probability of being dealt a one pair is\\
		\colorbox{yellow}{
			${P(E)=\frac{|E|}{|S|}=\frac{13\times6\times48\times47\times46\times \frac{1}{6} \times 5!}{311875200}=0.519}$
		}
	
		%8.c
		\item
		\mysolu
		For the number of elements of the event ${E=\{\text{two pair in 5 cards}\}}$, we can select 2 numeric value from 13 numbers and pick up 2 cards in the total 4 of each value. Then we can add the last card from the rest 44 cards. After all the 5 cards are selected, we can calculate the permutation of the 5 cards.\\
		${|E|={13 \choose 2}\cdot{4 \choose 2}^2 \cdot44\cdot 5!}$\\
		\myansw
		The probability of being dealt a two pair is\\
		\colorbox{yellow}{
			${P(E)=\frac{|E|}{|S|}=\frac{\frac{13\times12}{2}\times6^2\times44\times 5!}{311875200}=0.0475}$
		}
	
		%8.d
		\item
		\mysolu
		For the number of elements of the event ${E=\{\text{three of a kind in 5 cards}\}}$, we can select 1 numeric value from 13 numbers and pick up 3 cards in the total 4. Then we can add the last 2 cards from the rest 48 cards. After all the 5 cards are selected, we can calculate the permutation of the 5 cards.\\
		${|E|=13\cdot{4 \choose 3} \cdot{48 \choose 2}\cdot 5!}$\\
		\myansw
		The probability of being dealt three of a kind is\\
		\colorbox{yellow}{
			${P(E)=\frac{|E|}{|S|}=\frac{13\times4\times48\times47\times \frac{1}{2} \times 5!}{311875200}=0.0226}$
		}
		
		%8.e
		\item
		\mysolu
		For the number of elements of the event ${E=\{\text{four of a kind in 5 cards}\}}$, we can select 1 numeric value from 13 numbers and pick up all the cards. Then we can add the last 1 cards from the rest 48 cards. After all the 5 cards are selected, we can calculate the permutation of the 5 cards.\\
		${|E|=13\cdot 48\cdot 5!}$\\
		\myansw
		The probability of being dealt four of a kind is\\
		\colorbox{yellow}{
			${P(E)=\frac{|E|}{|S|}=\frac{13\times48 \times 5!}{311875200}=0.000240}$
		}		
		

	\end{enumerate}
	\item
	\begin{enumerate}
		%9.a
		\item
		Each die rolling have 6 possible resluts, the number of element in sample set\\
		${|S| = 6^6}$\\
		For the event ${E=\{\text{three different numbers and twice each}\}}$, we can choose 3 different numbers from 6 numbers and then put those 3 item gourp, which has 2 element in each, into 6 buckets.\\
		${|E|={6 \choose 3}{6 \choose 2,2,2}=1800}$\\
		\myansw
		The possibility that we will roll three different numbers, twice each is\\
		\colorbox{yellow}{
			${P(E)=\frac{|E|}{|S|}=\frac{1800}{6^6}=0.0386}$
		}
		%9.b
		\item
		For the event ${E=\{\text{some number exactly 4 times}\}}$, we can choose 1 number from 6 numbers and then choose the other 2 from the rest of 5 numbers. After choose all the numbers, put them into 6 bucket.\\
		${|E|={6 \choose 1}{5 \choose 2}{6 \choose 4, 1, 1}=1800}$\\
		\myansw
		The possiblility that we will roll some number exacly 4 times is\\
		\colorbox{yellow}{
			${P(E)=\frac{|E|}{|S|}=\frac{1800}{6^6}=0.0386}$
		}
	\end{enumerate}
	\item
	\begin{enumerate}
		
		\item
		\mysolu
		The sample space size equals to the number of all permutations of n integers, ${|S|=n!}$. The event space is the permutations that make the BSD have completely degenerate structure. Since we can only choose the maximum or the minimum from the rest of all the numbers for each node in sequence. The event space size is ${|E|=2^{n-1}}$\\
		\myansw
		The possibility that the resulting BST will have completely degenerate structure is\\
		\colorbox{yellow}{
			${P(E)=\frac{|E|}{|S|}=\frac{2^{n-1}}{n!}}$
			
		}\\
	
		\item
		\mysolu
		Usually, we can just use program to calculate the possibility of forming a completely degenerate BST of each ${n}$ and find the first one have the possibility lower than 0.01. Without help of computer, we can still estimate it. First, ${n!}$ must higher than 100, then ${n\geqslant6}$, ${P(E) = 0.0127}$  when ${n = 7}$,${P(E) = 0.0032}$ when ${ n = 8}$.\\
		\myansw
		The smallest value of ${n}$ for which the probability of forming a completely degenerate BST is less than 0.01 is\\
		\colorbox{yellow}{
			${n = 8}$ while ${P(E) = 0.0032}$
		}\\
		
		
	\end{enumerate}
	\item
	\begin{enumerate}
		\item
		\mysolu
		The sample space is all the distinct password permutations, then ${|S|=n!}$. If the first successful login will be exactly on her ${k}$-th try, that means the event space is the rest ${(n-1)}$ distinct password permutation. The event space size is ${|E|=(n-1)!}$.\\
		\myansw
		The probability that her first successful login will be exactly on her ${k}$-th try is\\
		\colorbox{yellow}{
			${P(E)=\frac{|E|}{|S|}=\frac{(n-1)!}{n!}}$
		}
		
	\end{enumerate}

%	\item A fraction with a binomial: $\frac{{x \choose y}}{z^2}$
%	\item A summation: $\sum\limits_{i=1}^n i^2$
	
%	\item A product in math mode: \[\prod\limits_{i=1}^n x = x^n\]
	
%	\item And a line \\ break.


\end{enumerate}

\newpage


\end{document}