%CS 109 Problem Set Xiaoqi Zhou


\documentclass{article}
	% basic article document class
	% use percent signs to make comments to yourself -- they will not show up.

\usepackage{amsmath}
\usepackage{amssymb}
	% packages that allow mathematical formatting

\usepackage{graphicx}
\usepackage{float}
	% package that allows you to include graphics

\usepackage[top=1in, bottom=1in, left=1in, right=1in]{geometry}

\frenchspacing
	% one space after periods

\usepackage{fancyhdr}
	% allows custom headers

\pagestyle{fancy}

\lhead{CS 109, Stanford University \\ Problem Set \#2} 
\rhead{Xiaoqi Zhou (xqzhou@stanford.edu) \\ 06237147}
\usepackage{color}
\usepackage{courier}
\cfoot{\thepage}
\renewcommand{\footrulewidth}{0.4pt} 
	%footer
\newcommand{\myansw}{\textbf{Answer:}\\}
\newcommand{\mysolu}{\textbf{Solution:}\\}
\begin{document}
\thispagestyle{fancy} %shows header/footer

\begin{enumerate}
	\item
	\begin{enumerate}
		%1.a
		\item
		\mysolu
		We calculate after how many rounds, the game provide will use up all their money.\\
		${m = \lfloor log_{2}X \rfloor}$\\
		Then we can derive the expectation of payoff of the game by considering 2 parts, normal payoff and the limited payoff after provider run out of money, after you win more or equal to ${2^m}$, no matter what happend, you will win ${X}$\\
		${E[Y] = \sum\limits_{n = 0}^{m}(\frac{1}{2})^{n+1}2^n+\frac{1}{2}^{n+1} X= (m+1)  \frac{1}{2} + \frac{1}{2}^{m+1} X}$\\
		\myansw
		${m =2}$\\
		\colorbox{yellow}{
			${E[Y] = 3 \times  \frac{1}{2}+ \frac{1}{2}^{3} \times 5=2.125}$
		}\\
		\item
		\myansw
		${m =8}$\\
			\colorbox{yellow}{
			${E[Y] = 9 \times  \frac{1}{2}+ \frac{1}{2}^{8} \times 500=5.477}$
		}\\
	
		\myansw
		${m =12}$\\
		\colorbox{yellow}{
			${E[Y] = 13 \times  \frac{1}{2}+ \frac{1}{2}^{13} \times 4096=7}$
		}\\

	\end{enumerate}

	\item
	\mysolu
	The expectation of new users in 5 minutes is ${E(X) = 2}$.
	The expectation of gain for the trip can be preseted as\\
	${E[(X+1) \times \$6 - \$7]=E[X]\times \$6 -\$1}$\\
	\myansw
	The expectation of Lyft to make in this trip is\\
	\colorbox{yellow}{
		${E[X] \times \$ 6 - \$1 = \$11}$
	}\\
	\item
	\begin{enumerate}
		\item
		\mysolu
		The possibility that ${x}$ bit are corrupted is\\
		${P(X=x)={2n \choose x}p^x(1-p)^{(2n-x)}}$\\
		\myansw
		The probability that the message ${ss}$ is received without any corruption is\\
		\colorbox{yellow}{${P(X=0)=(1-0.05)^8=0.663}$}\\
		\item 
		\mysolu
		For each first string ${s}$ which has any corruption bit, if the second time the corruption bit(s) are exactly the same, we can not detect that problem\\
		\myansw
		The probability that we can not detect the corrupted bit(s)\\
		\colorbox{yellow}{$P(E) = \sum\limits_{x = 1}^n {n \choose x}(p^x(1-p)^{(n-x)})^2$}\\
		\colorbox{yellow}{${P(E) = \sum\limits_{x = 1}^4 {4 \choose x}(0.05^x(1-0.05)^{(4-x)})^2=0.0074}$}\\
		\item
		\mysolu
		The probability that recipient can detect the corruption took place is\\
		${P(F) = 1 - P(X=0) - P(E)}$\\
		\myansw
		\colorbox{yellow}{${P(F) = 1 - (1-p)^{2n} - \sum\limits_{x = 1}^n {n \choose x}(p^x(1-p)^{(n-x)})^2 = 1 - 0.663 - 0.0074 = 0.329}$}\\
		

	\end{enumerate}
	\item
	There are 2 cases that the jury renders a correct decision, votes guilty when the defendants are actually guilty or votes innocent and the defendants are actally innocent.\\
	The probability of the defendants are actually guilty is\\
	${P(G) = 0.75}$\\
	For each juror, the probability that votes guilty ${P(H)}$ in different conditions is\\
	${P(H|G^c) = 0.1}$\\
	${P(H^c|G) = 0.2}$\\
	The probability that the jury decides the defendants are guilty is ${P(F)}$\\ 
	Then the probability make correct decision when the defenant is actually guilty (less than 4 juror votes innocent) is\\
	${P(F|G) = \sum\limits_{n = 0}^3 {12 \choose i} P(H^c|G)^i P(H|G)^{12-i}}$\\
	${P(F|G) = \sum\limits_{n = 0}^3 {12 \choose i} 0.2^i 0.8^{12-i}= 0.7946}$\\
	${P(FG) = P(F|G)P(G) = 0.5959}$\\
	Then the probability make correct decision when the defenant is actually innocent ( 1 - P(less than 4 juror votes innocent)) is\\
	${P(F^c|G^c) = 1 - \sum\limits_{n = 0}^4 {12 \choose i} P(H^c|G^c)^i P(H|G^c)^{12-i}}$\\
	${P(F|G) =1 -  \sum\limits_{n = 0}^3 {12 \choose i} 0.92^i 0.1^{12-i}= 1 - 3.4 \times 10^{-6}=1}$\\
	${P(FG) = P(F|G)P(G) = 0.5959}$\\
	

%	\item A product in math mode: \[\prod\limits_{i=1}^n x = x^n\]

\end{enumerate}

\newpage


\end{document}