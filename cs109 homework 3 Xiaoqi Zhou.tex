%CS 109 Problem Set Xiaoqi Zhou


\documentclass{article}
	% basic article document class
	% use percent signs to make comments to yourself -- they will not show up.

\usepackage{amsmath}
\usepackage{amssymb}
	% packages that allow mathematical formatting

\usepackage{graphicx}
\usepackage{float}
	% package that allows you to include graphics

\usepackage[top=1in, bottom=1in, left=1in, right=1in]{geometry}

\frenchspacing
	% one space after periods

\usepackage{fancyhdr}
	% allows custom headers

\pagestyle{fancy}

\lhead{CS 109, Stanford University \\ Problem Set \#2} 
\rhead{Xiaoqi Zhou (xqzhou@stanford.edu) \\ 06237147}
\usepackage{color}
\usepackage{courier}
\cfoot{\thepage}
\renewcommand{\footrulewidth}{0.4pt} 
	%footer
\newcommand{\myansw}{\textbf{Answer:}\\}
\newcommand{\mysolu}{\textbf{Solution:}\\}
\begin{document}
\thispagestyle{fancy} %shows header/footer

\begin{enumerate}
	\item
	\begin{enumerate}
		%1.a
		\item
		\mysolu
		We calculate after how many rounds, the game provide will use up all their money.\\
		${m = log_{2}X}$\\
		Then we can derive the expectation of payoff of the game by considering 2 parts, normal payoff and the limited payoff after provider run out of money\\
		${E(Y) = \sum\limits_{n = 1}^{m}(\frac{1}{2})^{n+1}2^n+\frac{1}{2}^{n+1} X= m  \frac{1}{2} + \frac{1}{2}^{n+1} X}$\\
		\myansw
		${E(Y) = m \part{title} \frac{1}{2}+ \frac{1}{2}^{n+1} X}$


	\end{enumerate}


%	\item A product in math mode: \[\prod\limits_{i=1}^n x = x^n\]



\end{enumerate}

\newpage


\end{document}