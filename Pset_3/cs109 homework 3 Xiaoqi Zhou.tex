%CS 109 Problem Set Xiaoqi Zhou


\documentclass{article}
	% basic article document class
	% use percent signs to make comments to yourself -- they will not show up.

\usepackage{amsmath}
\usepackage{amssymb}
	% packages that allow mathematical formatting

\usepackage{graphicx}
\usepackage{float}
	% package that allows you to include graphics

\usepackage[top=1in, bottom=1in, left=1in, right=1in]{geometry}

\frenchspacing
	% one space after periods

\usepackage{fancyhdr}
	% allows custom headers

\pagestyle{fancy}

\lhead{CS 109, Stanford University \\ Problem Set \#3} 
\rhead{Xiaoqi Zhou (xqzhou@stanford.edu) \\ 06237147}
\usepackage{color}
\usepackage{courier}
\cfoot{\thepage}
\renewcommand{\footrulewidth}{0.4pt} 
	%footer
\newcommand{\myansw}{\textbf{Answer:}\\}
\newcommand{\mysolu}{\textbf{Solution:}\\}
\begin{document}
\thispagestyle{fancy} %shows header/footer

\begin{enumerate}
	\item
	\begin{enumerate}
		%1.a
		\item
		\mysolu
		We calculate after how many rounds, the game provide will use up all their money.\\
		${m = \lfloor log_{2}X \rfloor}$\\
		Then we can derive the expectation of payoff of the game by considering 2 parts, normal payoff and the limited payoff after provider run out of money, after you win more or equal to ${2^m}$, no matter what happend, you will win ${X}$\\
		${E[Y] = \sum\limits_{n = 0}^{m}(\frac{1}{2})^{n+1}2^n+\frac{1}{2}^{n+1} X= (m+1)  \frac{1}{2} + \frac{1}{2}^{m+1} X}$\\
		\myansw
		${m =2}$\\
		\colorbox{yellow}{
			${E[Y] = 3 \times  \frac{1}{2}+ \frac{1}{2}^{3} \times 5=2.125}$
		}\\
		\item
		\myansw
		${m =8}$\\
			\colorbox{yellow}{
			${E[Y] = 9 \times  \frac{1}{2}+ \frac{1}{2}^{8} \times 500=5.477}$
		}\\
	
		\myansw
		${m =12}$\\
		\colorbox{yellow}{
			${E[Y] = 13 \times  \frac{1}{2}+ \frac{1}{2}^{13} \times 4096=7}$
		}\\

	\end{enumerate}

	\item
	\mysolu
	The expectation of new users in 5 minutes is ${E(X) = 2}$.
	The expectation of gain for the trip can be preseted as\\
	${E[(X+1) \times \$6 - \$7]=E[X]\times \$6 -\$1}$\\
	\myansw
	The expectation of Lyft to make in this trip is\\
	\colorbox{yellow}{
		${E[X] \times \$ 6 - \$1 = \$11}$
	}\\
	\item
	\begin{enumerate}
		\item
		\mysolu
		The possibility that ${x}$ bit are corrupted is\\
		${P(X=x)={2n \choose x}p^x(1-p)^{(2n-x)}}$\\
		\myansw
		The probability that the message ${ss}$ is received without any corruption is\\
		\colorbox{yellow}{${P(X=0)=(1-0.05)^8=0.663}$}\\
		\item 
		\mysolu
		For each first string ${s}$ which has any corruption bit, if the second time the corruption bit(s) are exactly the same, we can not detect that problem\\
		\myansw
		The probability that we can not detect the corrupted bit(s)\\
		\colorbox{yellow}{$P(E) = \sum\limits_{x = 1}^n {n \choose x}(p^x(1-p)^{(n-x)})^2$}\\
		\colorbox{yellow}{${P(E) = \sum\limits_{x = 1}^4 {4 \choose x}(0.05^x(1-0.05)^{(4-x)})^2=0.0074}$}\\
		\item
		\mysolu
		The probability that recipient can detect the corruption took place is\\
		${P(F) = 1 - P(X=0) - P(E)}$\\
		\myansw
		\colorbox{yellow}{${P(F) = 1 - (1-p)^{2n} - \sum\limits_{x = 1}^n {n \choose x}(p^x(1-p)^{(n-x)})^2 = 1 - 0.663 - 0.0074 = 0.329}$}\\
		

	\end{enumerate}
	\item
	\mysolu
	There are 2 cases that the jury renders a correct decision, votes guilty when the defendants are actually guilty or votes innocent and the defendants are actally innocent.\\
	The probability of the defendants are actually guilty is\\
	${P(G) = 0.75}$\\
	For each juror, the probability that votes guilty ${P(H)}$ in different conditions is\\
	${P(H|G^c) = 0.1}$\\
	${P(H^c|G) = 0.2}$\\
	The probability that the jury decides the defendants are guilty is ${P(F)}$\\ 
	Then the probability make correct decision when the defenant is actually guilty (less than 4 juror votes innocent) is\\
	${P(F|G) = \sum\limits_{n = 0}^3 {12 \choose i} P(H^c|G)^i P(H|G)^{12-i}}$\\
	${P(F|G) = \sum\limits_{n = 0}^3 {12 \choose i} 0.2^i 0.8^{12-i}= 0.7946}$\\
	${P(FG) = P(F|G)P(G) = 0.5959}$\\
	Then the probability make correct decision when the defenant is actually innocent ( 1 - P(less than 4 juror votes innocent)) is\\
	${P(F^c|G^c) = 1 - \sum\limits_{n = 0}^4 {12 \choose i} P(H^c|G^c)^i P(H|G^c)^{12-i}}$\\
	${P(F^c|G^c) =1 -  \sum\limits_{n = 0}^4 {12 \choose i} 0.9^i 0.1^{12-i}= 1 - 3.4 \times 10^{-6}=1}$\\
	${P(F^c) = P(F^c|G^c)P(G^c) = 0.25}$\\
	\myansw
	The probability that the jury renders a correct decision is\\
	\colorbox{yellow}{${P(FG)+P(F^cG^c) = 0.5959 + 0.25 = 0.8459}$}\\
	The percentage of defendant found guilty by the jury is\\
	\colorbox{yellow}{${P(F) = P(FG)+P(FG^c) = 0.5959 + 0 = 59.6\%}$}\\
	
	\item
	\mysolu
	Define event\\
	${X_i = \{\text{A person computer crash }i\text{ times in the month}\}}$\\
	${E = \{\text{The patch has had an effect on the user's computer}\}}$\\
	Then, based on the Poisson distribution\\
	${P(X_2|E) = e^{-3}\frac{3^2}{2!}=0.224}$\\
	${P(X_2|E^c) = e^{-5}\frac{5^2}{2!}=0.084}$\\
	\myansw
	${P(X_2E) = P(X_2|E)P(E) = 0.224 \times 0.75 = 0.168}$\\
	The probability that the patch has had an effect on the user's computer is\\
	\colorbox{yellow}{${P(E|X_2) = \frac{P(X_2E)}{P(X_2)}=\frac{0.168}{0.224\times 0.75+0.084 \times 0.25}=0.889}$}\\	
	
	\item
	\mysolu
	${E[\sum\limits_{i = 1}^k X_i]=\sum\limits_{i=1}^k E[X_i]}$\\
	${E[X_i]=1\times P(X_i = 1) + 0 \times P(X_i = 0)=1 \times (1 - (1 - p_i)^n)+ 0 \times (1 - p_i)^n}$\\
	\myansw
	The expected number of buckets that have at least one string hashed into them is\\
	\colorbox{yellow}{${E[\sum\limits_{i = 1}^k X_i]=\sum\limits_{i=1}^k (1-(1-p_i)^n)}$}\\
	\item
	\mysolu
	The CDF of given uniformly random distribution is\\
	$F(a)=\left\lbrace \begin{array}{rcl}
		0 & &{a\leq 0}\\
		\frac{a}{n} & &{0<a<n}\\
		1 & & a\geq n\\
	\end{array} \right.$\\
	Only in the cases that ${x<0.2n}$ or ${x>0.8n}$, the shorter piece is less than 1/4th of the longer one.
	\myansw
	\colorbox{yellow}{${P(x<0.2n)+P(x>0.8n)=F(0.2n)+(1-F(0.8n))=\frac{0.2n}{n}+(1-\frac{0.8n}{n})=0.4}$}\\
	
	\item
	\begin{enumerate}
		\item
		\mysolu
		${1 = \int_{-\infty}^{+\infty}f(x)dx=\int_{-1}^{+1}c(3-2x^2)dx}$\\
		$1 = c(3x-\frac{2}{3}x^3) \left| \begin{array}{rcl}
			1\\
			-1\\
			\end{array}
			\right. = \frac{14}{3}c$\\
		\myansw
		\colorbox{yellow}{${c = \frac{3}{14}}$}\\
		\item
		\myansw
		${F(a) = P{X < a} = \int_{-\infty}^{a}f(x)dx}$\\
		$F(a)=\left\lbrace \begin{array}{rcl}
		0 & &{a\leq -1}\\
		(3a-\frac{2}{3}a^3)c-(-\frac{7}{3}c) & &{-1<a<1}\\
		1 & & a\geq 1\\
		\end{array} \right.$\\
		\colorbox{yellow}{
			${F(a) = \left\lbrace \begin{array}{rcl}
				0 & &{a\leq -1}\\
				(\frac{9}{14}a-\frac{1}{7}a^3)+\frac{1}{2} & &{-1<a<1}\\
				1 & & a\geq 1\\
				\end{array} \right.}$
		}\\
		\item
		\myansw
		${E[X] = \int_{-\infty}^{+\infty}xf(x)dx}$\\
		${E[X] = \int_{-1}^{+1}x\frac{3}{14}(3 - 2x^2)dx}$\\
		\colorbox{yellow}{${E[X] = \frac{9}{28}x^2 - \frac{3}{28}x^4\left| \begin{array}{rcl}
			1\\
			-1\\
			\end{array}
			\right. = 0}$}\\
		
	\end{enumerate}
	\item 
	\mysolu
	${P(A) = \alpha}$\\
	${P(B) = 1-\alpha}$\\
	Since, ${P(A) = P(B^c)}$\\
	${P(R) = P(RA)+P(RB)=P(R|A)P(A)+P(R|B)P(B)}$\\
	${P(R) = (f_A(x)dx|x = 5)\alpha + (f_B(x)dx|x = 5)(1-\alpha)}$\\
	${P(R|A) = \frac{1}{\sqrt{2\pi}\times3}e^{-(5-6)^2/(2\times 9)} dx = 0.1258 dx}$\\
	${P(R|B) = \frac{1}{\sqrt{2\pi}\times2}e^{-(5-4)^2/(2\times 4)} dx = 0.1760 dx}$\\
	${P(RA) = 0.1258 dx \, \alpha}$\\
	${P(RB) = 0.1760 (1 - \alpha)}$\\
	${P(R) = P(RA)+P(RB) = 0.1760 - 0.0502 \alpha \, dx}$\\
	\myansw
	${P(A|R) = \frac{P(AR)}{P(R)}=\frac{0.1258 \alpha \,dx }{0.1760 - 0.0502  \alpha \, dx}=\frac{0.1258 \alpha  }{0.1760 - 0.0502 \alpha}}$\\
	${\frac{ 0.1258 \alpha  }{0.1760 - 0.0502 \alpha}=0.5}$\\
	\colorbox{yellow}{
		${\alpha = 0.583}$
	}\\

	\item
	\begin{enumerate}
		\item
		\mysolu
		For each return number from 1 to 10 we can derive the probability\\
		$P(X=i) = 0.1$\\
		$P(X = -1) = 0$\\
		\myansw
		\colorbox{yellow}{$E[X] = \sum\limits_{i=0}^{9}0.1i =4.5$}\\
		\item
		\mysolu
		The only chance that the founction can return is when arr[mid] = key. So the result would as same as the previous\\
		$P(X=i) = 0.1 | 1\leq i \leq 10$\\
		\myansw
		\colorbox{yellow}{$E[X] = \sum\limits_{i=0}^{9}0.1i =4.5$}\\
	\end{enumerate}
	\item 
	\begin{enumerate}
		\item
		\mysolu
		Since the total hash trial of the strings ${3m = 72000}$ are very high and the probability for each string hashed into certain bucket is very low ${\frac{1}{8000}}$ . We can use binomial distribution function to derive the probability of each number of strings that hashed into a certain bucket.
		$P\{X=i\} \approx e^{- \lambda}\frac{\lambda^i}{i!}$\\
		${\lambda = 3m \frac{1}{n} = 72000 \times 
		\frac{1}{8000} = 9}$\\
		\myansw
		The probability that the first bucket has 0 strings hashed into it is\\
		\colorbox{yellow}{
			${P\{X = 0\} \approx e^{-9}\frac{9^0}{0!}=1.234\times 10^{-4}}$
		}\\
		\item
		\myansw
		The probability that the first bucket has 10 or fewer strings hashed to it is\\
		\colorbox{yellow}{
			${P\{X \leq 10\} \approx \sum\limits_{i=0}^{10} e^{-9}\frac{9^i}{i!}=0.706}$
		}
		\item
		\mysolu
		For a bloom filter which have ${X = i}$ bits are 0s, the probability ${P(E)}$ of a string that is reported in the set incorrectly is\\
		${P(E|\{X = i\})= (1-\frac{i}{n})^3}$\\
		For each bit, the probability that remains to 0s after 25000 strings added is\\
		$p= e^{-0.75}\frac{0.75^0}{0!} = 0.4724$\\
		Let $\lambda = np = 4.724\times 10^4$\\
		The probability that $X = i$ bucket have no string been hashed into is\\
		${P(\{X=i\}) =\frac{e^{-\lambda}\lambda^i}{i!}}$\\
		${P(E\{X=i\}) = \frac{(n-i)^3}{n^3} \frac{e^{-\lambda}\lambda^i}{i!}}$\\
		${P(E) = \sum\limits_{i=0}^{+\infty}P(E\{X=i\})=
			 \sum\limits_{i=0}^{+\infty}\frac{e^{-\lambda}\lambda^i}{i!}-
			 \frac{1}{n}\sum\limits_{i=0}^{+\infty}\frac{3ie^{-\lambda}\lambda^i}{i!}+
			 \frac{1}{n^2}\sum\limits_{i=0}^{+\infty}\frac{3i^2e^{-\lambda}\lambda^i}{i!}-
			\frac{1}{n^3} \sum\limits_{i=0}^{+\infty}\frac{i^3e^{-\lambda}\lambda^i}{i!}}$\\
		
		${P(E) =
			1-
			\frac{3}{n}E[X]+
			\frac{3}{n^2}E[X^2]-
			\frac{1}{n^3}E[X^3]}$\\
		${\frac{1}{n^3}E[X^3] = \frac{\lambda}{n^3}\sum\limits_{i=1}^{+\infty}\frac{i^2e^{-\lambda}\lambda^{(i-1)}}{(i-1)!}=\frac{\lambda}{n^3}\sum\limits_{i=1}^{+\infty}(\frac{(i-1)^2e^{-\lambda}\lambda^{(i-1)}}{(i-1)!}+\frac{2(i-1)e^{-\lambda}\lambda^{(i-1)}}{(i-1)!}+\frac{e^{-\lambda}\lambda^{(i-1)}}{(i-1)!})}$\\
		${\frac{1}{n^3}E[X^3] = \frac{\lambda}{n^3}(E[X^2] + 2E[X]-1)=\frac{\lambda}{n^3}(\lambda(\lambda+1) + 2\lambda+1)=\frac{1}{n^3}(\lambda^3+3\lambda^2+\lambda)}$\\
		${P(E) = 1 - \frac{3}{n}\lambda + \frac{3}{n^2}\lambda(\lambda+1) - \frac{1}{n^3}(\lambda^3+3\lambda^2+\lambda)}$\\
		${P(E) \approx 1 - \frac{3}{n}\lambda + 3(\frac{\lambda}{n})^2 -  (\frac{\lambda}{n})^3= 1 - 3p + 3p^2 - p^3}$\\
		\myansw
		\colorbox{yellow}{${P(E) \approx 0.147}$}\\
		The anwser is as same as the simple way that just use the probability that all the 3 hash functions hits the bits changed to 1s.\\
		${P(E) = (1-\frac{E[X]}{n})^3 = (1-p)^3}$
		\item
		\mysolu
		For each bit, the probability that remains to 0s after 25000 strings added is\\
		$p= e^{-0.25}\frac{0.25^0}{0!} = 0.779$\\
		\myansw
		The new string that will be incorrectly reported as in the set is\\ 
		\colorbox{yellow}{${P(E) \approx 1 - p = 0.221}$}\\
		That incorrect rate is higher than using 3 hash function. We need to find a balance between the bit-array size vs. the efficiency.\\

		
		
	\end{enumerate}
	\item
	\begin{enumerate}
		%12.a
		\item
		\mysolu
		Since all the polls are equivalent, we can just add all the samples that vote A and divide by the total sample number.\\
		\myansw
		The probability that a random person in France votes for candidate A is\\
		\colorbox{yellow}{$P(A) = \frac{4881}{7453}=0.655$}\\
		\item
		\mysolu
		Since the total number of people is too large, we can use 64888 total population and iterate the experiment for 10000 times.\\
		The result is all the 10000 experiment show that A wins.\\
		\myansw
		\colorbox{yellow}{$P(A) \approx 1.00$}\\
		The distribution of votes for A is a Poisson distribution with $\lambda = np $\\
		Another way to solve this problem is, we can treat the distribution of votes for A as a normal distribution which has $\mu = np$,$\sigma^2 = np(1-p)$ \\
		$F(0.5n) = \Phi(\frac{0.5n - np}{\sqrt{np(1-p)}}) = 1 - \Phi(\frac{0.155n}{0.475\sqrt{n}})$\\
		Since $\lambda$ is very large\\
		$\Phi(0.326\sqrt{n})=0$\\
		\colorbox{yellow}{$P(A)=F(0.5n) \approx 1$}\\
	\end{enumerate}
	\item
	\begin{enumerate}
		%13.a
		\item
		\myansw
		\colorbox{yellow}{$\mu_A = 5.324$}\\
		\colorbox{yellow}{$\mu_B = 2.926$}\\
		\colorbox{yellow}{$\sigma = 4.0542$}\\
		\colorbox{yellow}{$\sigma^2 = 16.436$}\\
		%13.b
		\item
		\myansw
		The simulation result shows\\
		\colorbox{yellow}{${P(A) = 0.662}$}\\
		\item
		\myansw
		If we believe that the poll results can represent the probability in the entire population precisely, the result of 12(b) is reasonable that candidate A always wins. But in the real life the limited quantity of samples can not reflect the actual distribution in the entire population.\\
		The reason why we should use 13(b) model is that model considered the inconsistence among the polls and difference from the poll to the actual votes. The polls have very limited sample volume, all of them can not represent the true distribution very accurately.
		\item
		\myansw
		Even the result is candidate B wins, it is not means part(b) is wrong. Because even in the simulation there are 33.9\% of total runs shows that candidate B wins. That can happen in real life.
		The probability can only tell us the distribution of an event happens lot of times, but can not decide one particular trial.
		\item
		\myansw
		The new model I create is\\
		${S_A \sim N(\mu = \frac{1}{N}\sum\limits_{i=1}^N p_{Ai}, \sigma^2  = \frac{1}{N} \sum\limits_{i = 1}^N (p_Ai - \mu)^2\times 4.5)}$\\
		${S_B \sim N(\mu = \frac{1}{N}\sum\limits_{i=1}^N p_{Bi}, \sigma^2  = \frac{1}{N} \sum\limits_{i = 1}^N (p_Bi - \mu)^2\times 4.5)}$\\
		${P(A) = P(S_A > S_B)}$\\
		The simulation results show\\
		For poll.csv ${P(A) = 0.688}$\\
		For extraPolls1.csv ${P(A) = 0.693}$\\
		For extraPolls2.csv ${P(A) = 0.656}$\\
		
		
		
	\end{enumerate}

\end{enumerate}

\newpage
I found there were some mistake in the question 12.b and updated my answers in the night of 5/3.


\end{document}